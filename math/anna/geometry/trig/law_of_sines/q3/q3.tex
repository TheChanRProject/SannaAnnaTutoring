\documentclass{article}
\usepackage{amsmath}
\begin{document}
\author{Ana Bhattacharjee}
\title{Law of Sines: Question 3}
\date{\today}

\begin{center}
In this problem we are asked the following:
\begin{enumerate}
  \item Use the properties of right triangles and triangle ABC to prove the Law of Sines.
  \item Find the length of BC, rounded to the nearest tenth of a unit.
\end{enumerate}
\par
We can first start by looking at the two right triangles given to us ABD and BDC. The side opposite to the $46^{\circ}$ can be found using sine. Similarly, the side opposite to the $31^{\circ}$ in the other triangle can be found using sine.
\begin{align}
sin(46) = \frac{x}{17} \\
sin(31) = \frac{x}{a} \\
x = 17 sin(46) \\
sin(31) = \frac{17 sin(46)}{a} \\
a sin(31) = 17 sin(46) \\
\frac{sin(31)}{17} = \frac{sin(46)}{a}
\end{align}
To find the length of BC, we use the law of sines.
\begin{align}
  a = \frac{17 sin(46)}{sin(31)} \\
  a \approx 24^{\circ}
\end{align}
\end{center}
\end{document}
