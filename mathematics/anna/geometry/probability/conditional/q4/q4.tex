\documentclass{article}
\usepackage{amsmath}
\begin{document}
\title{Question 4}
\author{Ana Bhattacharjee}
\date{\today}
\maketitle

\begin{center}

  \begin{align}
    P(W) = \frac{8}{19} \\
    P(R) = \frac{11}{18}
  \end{align}
  The probability of $P(R | W)$ is the probability of plucking a red rose the second draw given that the first rose plucked was a white rose. This probability is calculated as:
  \begin{align}
    P(R | W) = \frac{P(R) * P(W)}{P(W)} \rightarrow P(R)
  \end{align}
  The probability of $P(W | R)$ is the probability of plucking a white rose given the red rose has been plucked in the second draw. This probability is calculated as:
  \begin{align}
    P(W | R) = \frac{P(W) * P(R)}{P(R)} \rightarrow P(W)
  \end{align}
  The values of these two conditional probabilites are not the same. 
\end{center}
\end{document}
