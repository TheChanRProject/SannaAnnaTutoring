\documentclass{article}
\usepackage{amsmath}
\begin{document}
\title{Quiz: Question 9}
\author{Ana Bhattacharjee}
\date{\today}
\maketitle

\begin{center}
  \begin{align}
    P(7) = \frac{6}{36} \\
    P(T - 7) = \frac{30}{36} \\
  \end{align}
  Based on the probabilities, one has a much higher chance of paying \$500 under the insurance plan since the probability of rolling a 7 is $\frac{6}{36}$. Since the probabilities of the two events are not equal or even close to being equal, the deal is, by definition, not a fair one, but at the same time, it is better on the player that it is not fair. The reason for this is because if there was an equal probability between the two options in the insurance plan, one can be much more hesitant on whether or not to accept the insurance in the first place. 
\end{center}
\end{document}
