\documentclass{article}
\usepackage{amsmath}
\begin{document}
\title{Question 11}
\author{Ana Bhattacharjee}
\date{\today}
\maketitle

\begin{center}
We're finding missing counts. The counts will be expressed similarly to the conditional probability expression.
  \begin{align}
    P(\text{Girl} | \text{Lone Ranger}) = 100 - 50 \rightarrow 50 \\
    P(\text{Sesame Street}) = 30 + 80 \rightarrow 110 \\
    P(\text{Simpsons} | \text{Girl}) = 200 - 130 \rightarrow 70 \\
    P(\text{Boy} | \text{Simpsons}) = 90 - 70 \rightarrow 20
  \end{align}
\end{center}

\begin{table}[ht]
\vspace{-1.5em}
\centering
\begin{tabular}{|l|l|l|l|l|}
\hline
 & Lone Ranger & Sesame Street & Simpsons & Total \\ \hline
Boys & 50 & 30 & 20 & 100 \\ \hline
Girls & 50 & 80 & 70 & 200 \\ \hline
Total & 100 & 110 & 90 & 300 \\ \hline
\end{tabular}
\end{table}


\end{document}
