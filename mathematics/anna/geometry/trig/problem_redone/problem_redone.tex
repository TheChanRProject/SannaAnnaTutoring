\documentclass{article}
\usepackage{amsmath}
\usepackage{graphicx}
\begin{document}
\title{Problem Redone: Question 8}
\author{Ana Bhattacharjee}
\date{\today}
\maketitle

\begin{center}
  \begin{figure}[!htbp]
    \includegraphics[width=0.90\columnwidth]{new_image}
    \caption{Land}
  \end{figure}
  Since the three isoceles triangles are congruent, we can define each isoceles triangle with the following dimensions:
  \begin{figure}[!htbp]
    \includegraphics[width=0.90\columnwidth]{triangle}
    \caption{Isoceles Triangle}
  \end{figure}
  Now that we have three equivalent isoceles triangles, we know that the farmer wants to plant vegetation in the area containing two of these triangles. In order to do this, we need to use the formula $ A = \frac{1}{2}ab*sin(C)$ to find the area of one isoceles triangle. Since the area of each green isoceles triangle will be the same, we can simply do the following:
  \begin{align}
    A = 2*\frac{1}{2} ab*sin(C) \\
    A = ab*sin(C) \rightarrow 185 (283) sin(80) \\
    A = 52355 sin(80) \\
    A \approx 51559.6 \text{ yd}^2
  \end{align}
  \par
  In order to find the cost for the farmer to cover the vegetation part of the field with top soil, we first have to divide the area of the vegetation by the area that each bag of top soil covers. Then we multiply the number of bags found with the cost of each bag.
  \begin{align}
    n_{\text{bags}} = \frac{A_{\text{vegetation}}}{A_{\text{bag}}} \\
    n_{\text{bags}} = \frac{51559.6}{36} \approx 1432 \text{bags} \\
    \text{cost} = \$ 12 * n_{\text{bags}} \rightarrow 12 * 1432 \\
    \text{cost} = \$ 17,184
  \end{align}

\end{center}

\end{document}
