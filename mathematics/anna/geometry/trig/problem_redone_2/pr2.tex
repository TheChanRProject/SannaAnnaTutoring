\documentclass{article}
\usepackage{amsmath}
\begin{document}
\title{Problem Redone}
\author{Ana Bhattacharjee}
\date{\today}
\maketitle
\begin{center}
  \begin{enumerate}
    \item Find the total perimeter of the grazing area.
    \item If the cost of the fence is \$ 7.95 per linear foot, how much will it cost to place a fence around the entire grazing area?
    \item Suppose the grazing area is 1/60 of an industrial grazing area for a major industry farm. What is the total cost to build a fence around the entire land of the large scale farm?
  \end{enumerate}
  In order to find the perimeter of the grazing area, we must use the area formula for a regular polygon.
  \begin{align}
    A = \frac{1}{2}aP \\
    500 = \frac{1}{2}(10)P \\
    P = \frac{500}{5} = 100 \text{ft}
  \end{align}
  Since the perimeter is 100 ft, we can now multiply this number by the cost per linear foot to get the cost of the fence around the grazing area.
  \begin{align}
    C = \$7.95 * P \\
    C = \$7.95 * 100 \\
    C = \$795
  \end{align}
  In order to find the cost, we first need to find the area of the large industrial farm. To do this, we multiply the small grazing area's area by 60 to get the large scale area. Once we do this, we have to find the new perimeter. Finally, we multiply the resulting perimeter by the cost per linear foot.
  \begin{align}
    A = 60 * 500 \\
    A = 30000 \text{ft}^2 \\
    A = \frac{1}{2}aP \\
    30000 = \frac{1}{2} (10) P \\
    P = \frac{30000}{5} = 6000 \text{ft} \\
    C = \$7.95 * 6000 \\
    C = \$47,700
  \end{align}
\end{center}
\end{document}
